\documentclass{article}
\usepackage[utf8]{inputenc}
\usepackage[letterpaper]{geometry}
\usepackage{setspace}
% \geometry{verbose,tmargin=.7in,bmargin=.7in,lmargin=.7in,rmargin=.7in}

% FONTS
\usepackage{color}
\usepackage{charter}
\usepackage[T1]{fontenc}
\usepackage{amsfonts}         % HD AMS fonts
\usepackage{bm}

\usepackage[english]{babel}
\usepackage{graphicx}

% MATH SYMBOLS / ENVIRONMENTS
\usepackage{amsmath}
\usepackage{amssymb}
\usepackage{amsthm}
\usepackage{stmaryrd}

% DEFINITION
% Variables
\newcommand{\eps}{\epsilon}
\newcommand{\veps}{\varepsilon}
\newcommand{\vphi}{\varphi}

% Fonts / scripts
\newcommand{\mb}[1]{\mathbf{#1}}
\newcommand{\mc}[1]{\mathcal{#1}}
\newcommand{\mf}[1]{\mathfrak{#1}}
\newcommand{\bb}[1]{\mathbb{#1}}
\newcommand{\mr}[1]{\mathrm{#1}}

% Abbreviations
\newcommand{\st}{\text{s.t. }}
\newcommand{\wrt}{\text{w.r.t. }}
\newcommand{\iid}{\text{i.i.d. }}
\newcommand{\whp}{\text{w.h.p. }}

% Operators
\newcommand{\diag}{\text{diag}}
\newcommand{\Exp}{\text{Exp}}
\newcommand{\grad}{\text{grad}}
\newcommand{\prox}{\text{prox}}
\newcommand{\proj}{\text{proj}}
\newcommand{\Log}{\text{Log}}
\newcommand{\supp}{\text{supp}}
\newcommand{\Tr}{\text{Tr}}
\newcommand{\tsum}{\textstyle \sum}
\newcommand{\conv}{\circledast}

% Brackets
\newcommand{\abs}[1]{\left\lvert #1\right\rvert }
\newcommand{\inner}[1]{\left\langle #1\right\rangle }
\newcommand{\iter}[1]{^{(#1)}}
\newcommand{\norm}[1]{\left\lVert #1\right\rVert }
\newcommand{\paren}[1]{\left(  #1\right)  }
\newcommand{\brac}[1]{\left[  #1\right]  }
\newcommand{\set}[1]{\left\{  #1\right\}  }

% Misc.
\newcommand{\latex}{\mr{L^AT_EX}}
\newcommand{\indep}{\rotatebox[origin=c]{90}{$\models$}}
\newcommand{\udset}[2]{\underset{\text{#1}}{\underbrace{#2}}}

% Numbering, references, citations
\numberwithin{equation}{section}
\usepackage{hyperref}
\usepackage[nameinlink]{cleveref}
\hypersetup{
    colorlinks=true,%
    citecolor=blue,%
    filecolor=blue,%
    linkcolor=blue,%
    urlcolor=blue
}

\title{The second-order geometry of \\ short-and-sparse blind deconvolution}
\author{Yenson Lau}

\begin{document}
\maketitle

\begin{abstract}
  These notes describe a set of experiments for interrogating the curvature of nonconvex objective functions for short-and-sparse blind deconvolution (SaS-BD) over the course of optimization.
\end{abstract}

\section{Introduction}
In SaS-BD, we observe a length-$m$ signal
\begin{equation*}
  \mb y \;\;=\;\; \iota\mb a_0 \;\conv\; \mb x_0 \;+\; \mb w,
\end{equation*}
the cyclic convolution between a {\em short} kernel $\mb a_0\in\bb S^{p_0-1}$, zero-padded to length $m$, and a {\em sparse} activation map $\mb x_0\in\bb R^m$, with possible additive noise $\mb w$. Our goal is to recover $\mb a_0$ and $\mb x_0$ up to some scaling or cyclic shift, as
\begin{equation*}
  \iota\mb a_0 \;\conv\; \mb x_0 \;\;\equiv\;\;
  \alpha\;s_\ell[\iota\mb a_0] \;\conv\; \alpha^{-1} s_{-\ell}[\mb x_0]
\end{equation*}
for any $\alpha\in\set{\bb R\setminus0}$ and $\ell\in\bb Z$. We refer to these as the scaling and shift symmetries respectively.

A natural approach is to begin by formulating SaS-BD as a nonconvex optimization problem,
\begin{equation} \label{eqn:obj-Psi}
  \min_{\mb a \in \bb S^{p-1}, \mb x} \;\;\Big[\;
    \Psi_\lambda(\mb a, \mb x) \;\doteq\; \psi(\mb a, \mb x ; \mb y)
    \;+\; \lambda\rho(\mb x)
  \;\Big],
\end{equation}
which minimizes a reconstruction error between $\mb y$ and $\mb a\conv\mb x$, plus a sparse penalty $\rho$ on $\mb x$. For example, combining a squared error loss with an $\ell_1$-norm penalty yields the {\em bilinear-lasso} (BL) formulation
\begin{equation} \tag{BL} \label{eqn:bl}
  \Psi_\lambda(\mb a,\mb x) \;\;=\;\; \tfrac12\norm{\mb a\conv\mb x-\mb y}_2^2 \;+\; \lambda\norm{\mb x}_1.
\end{equation}
A simplified problem, known as the {\em dropped-quadratic} formulation (DQ), can be obtained when the circulant matrix of $\mb a$, $\mb C_{\mb a}$, satisfies $\mb C_{\mb a}^T\mb C_{\mb a} \simeq \mb I$ over the course of optimization,
\begin{align} \tag{DQ} \label{eqn:dq}
  \Psi_\lambda(\mb a,\mb x) \;\;&=\;\; \tfrac12\norm{\mb x}_2^2 \;-\; \inner{\mb a\conv \mb x, \mb y} \;+\; \tfrac12\norm{\mb y}_2^2+\lambda\rho(\mb x), \\
  \rho(\mb x) \;\;&=\;\; \tsum_i \paren{x_i^2+\delta^2}^{\frac12}.
\end{align}

\subsection{Regional landscape geometry}
When $\mb x_0$ is generic, the objective function becomes regular on $\mb a$ after marginalization on $\mb x$,
\begin{equation} \label{eqn:obj-phi}
  \eqref{eqn:obj-Psi} \;\equiv\; \min_{\mb a \in \bb S^{p-1}} \;\Big[\;
    \vphi_\lambda(\mb a) \;\doteq\; \min_{\mb x}\; \Psi_\lambda(\mb a, \mb x)
  \;\Big].
\end{equation}
This phenomenon appears to be {\em largely independent} of the specific formulation used for SaS-BD, but finds a relatively simple expression in the \eqref{eqn:dq} formulation, for which is summarized as follows \cite{kuo2019geometry}:


\subsection{Sparsity-coherence tradeoff}


\section{Experimental outline}

{\small
\bibliographystyle{alpha}
\bibliography{deconv,ncvx,cdl}
}

\end{document}